\documentclass[14pt,xcolor=dvipsnames,pdftex]{beamer}
\usepackage[utf8x]{inputenc}
\usepackage[T1]{fontenc}
\usepackage[english]{babel}
\usepackage{hyperref}
\usepackage{pgfpages}
\usepackage{color}
\usepackage{textcomp}
\usepackage{alltt}
\usepackage{amsmath}
\usepackage{amsfonts}
\usepackage{booktabs}
\usepackage{textcomp}
\usepackage{color}

\usepackage{tikz}
\usetikzlibrary{calc,arrows,matrix,fit,positioning}

\usepackage{lmodern}

\newcommand{\tikzmark}[2]{\tikz[overlay,remember picture,baseline=(#1.base)] \node (#1) {#2};}

\usepackage{inconsolata} % TT font
\usepackage{array}
\usepackage{url}
%\usepackage[svgnames]{xcolor}\\
%\usetheme[secheader]{Boadilla}
%\usefonttheme{sans}
\setbeamersize{text margin left=1cm,text margin right=1cm}
\setbeamertemplate{section in toc}[Palo Alto]

\newcolumntype{T}{c<{\ttfamily}}

\title{Phylogenomic inference}
\author{Uli Köhler}
\date{February 3rd 2014}

\setbeamertemplate{footline}
{%
%\begin{beamercolorbox}[wd=0.5\textwidth,ht=3ex,dp=1.5ex,leftskip=.5em,rightskip=.5em]{author in head/foot}%
%\usebeamerfont{author in head/foot}%
%\insertframenumber\hfill\insertshortauthor%
%\end{beamercolorbox}%
\vspace*{-4.5ex}\hspace*{0.5\textwidth}%
\begin{beamercolorbox}[width=\textwidth,ht=3ex,dp=1.5ex,left]{title in head/foot}%
\usebeamerfont{title in head/foot}%
Folie \insertframenumber{} von \inserttotalframenumber\hspace{8mm}
%\insertshorttitle%
\end{beamercolorbox}%
}

\AtBeginSection[]{} % for optional outline or other recurrent slide

\begin{document}
\bgroup
\setbeamercolor{background canvas}{bg=black}

\begin{frame}[plain]{}
\end{frame}

\egroup

\frame{\titlepage}

\begin{frame}
\begin{itemize}
\item Issues of non-phylogenic functional prediction
\item Phylogenetic tree reconciliation
\item Phylogenomic databases and algorithms:
\begin{itemize}
 \item SIFTER
 \item PhyloFacts
\end{itemize}
\item Common problems of phylogenomic predictions
\item Future of phylogenomics
\item Seminar conclusion
\end{itemize}
\end{frame}

\begin{frame}{Non-phylogenomic function prediction}
 \begin{itemize}
  \item Intention: Predict gene/protein function
  \item \textit{High-throughput sequencing}\\
  \textrightarrow Many proteins, few information available
  \item Slow manual annotation: $~90000$ PDB structures vs $5.1\times10^6$ UniProt/TrEMBL sequences
  \item Difficult to separate \textit{orthologs} and \textit{paralogs}
 \end{itemize}
\end{frame}

\begin{frame}[allowframebreaks]{What is phylogenomics?}
 \begin{tikzpicture}[scale=0.65,>=latex',yshift=-3cm]
 \tikzstyle{toptext}=[block,font=\huge,minimum height=3em,anchor=mid,text centered]
 \tikzstyle{labeltext}=[block,font=\large,minimum height=3em,anchor=west,text centered]
 \begin{scope}[yshift=-4cm]
  % Text at the top
  \node [toptext](phylo) {\raisebox{0pt}[\height][0pt]{\textcolor{ForestGreen}{Phylo}}};
  \node [toptext,right=-3mm of phylo](genomic)
    {\raisebox{0pt}[\height][0pt]{\textcolor{Red}{genomic}}};
  \node [toptext,right=0cm of genomic] (inference)
    {\raisebox{0pt}[\height][0pt]{\textcolor{Blue}{inference}}};
  % Labels
  \node [labeltext,below=4cm of phylo,xshift=4cm] (phylolabel)
      {\textcolor{ForestGreen}{Evolutionary relationship (phylogenetics)}};
  \node [labeltext,below=2.5cm of genomic,xshift=1cm] (genomicslabel)
      {\textcolor{Red}{utilize genomic information}};
  \node [labeltext,below=10mm of inference] (inferencelabel)
      {\textcolor{Blue}{infer function}};
  % Arrows
  \draw [thick,->,arrowhead=4mm, line width=2pt]
      (phylolabel.north -| phylo.north) to (phylo);
  \draw [thick,->,arrowhead=4mm, line width=2pt]
      (genomicslabel.north -| genomic.north) to (genomic);
  \draw [thick,->,arrowhead=4mm, line width=2pt]
      (inferencelabel.north -| inference.north) to (inference);
 \end{scope}
 \end{tikzpicture}
 \framebreak
 \begin{itemize}
  \item Function prediction enhancement concept
  \item Can be applied to both genes and proteins
  \item Attempt to \textbf{separate \textit{orthologs} and \textit{paralogs}}
  \item Infer function based on homology
  \item \textit{Phylogenetic tree reconciliation}:\\
        Identify \textit{speciation} and \textit{duplication} events in phylogenetic trees
 \end{itemize}
\end{frame}

\begin{frame}<1-3>{Tree reconciliation}
\begin{tikzpicture}
\onslide<1->
\tikzset{
  protein/.style = {block, align=center, text centered,fill=orange,
		    draw, rectangle, rounded corners},
  ancestor/.style = {block,minimum size=0cm},
  connection/.style = {thick,draw,arrowhead=4mm, line width=2pt}
};
%Protein nodes
\node [protein] (proteinA) {A};
\node [protein, right=2cm of proteinA] (proteinB) {B};
\node [protein, right=2cm of proteinB] (proteinC) {C};
%Invisible ancestors
\node [ancestor,yshift=2cm] (AB) at ($(proteinA)!0.5!(proteinB)$) {};
\node [ancestor,yshift=3cm] (ABC) at ($(proteinB)!0.5!(proteinC)$) {};
%Paths
\draw [connection]
      (proteinA) |- (AB.east);
\draw [connection]
      (proteinB) |- (AB.west);
\draw [connection]
      ([yshift=1.4mm]AB.south) |- (ABC.east);
%To-root
\draw [connection]
      ([yshift=1.4mm]ABC.south) -- ([yshift=1cm]ABC.north);
%C -> root
\onslide<1-2>
\draw [connection]
      (proteinC) |- (ABC.west);
\onslide<3>
\draw [connection,dashed]
      (proteinC) |- (ABC.west);
%Text
\onslide<1>
\node [right=5mm of proteinC,text width=3cm,yshift=2cm] (txt)
      {Are B and C \textit{ortholog}\\or \textit{paralog} in respect to A?};
\onslide<2>
\node [left=0mm of ABC,yshift=10mm,x](dupspecLabel){\textcolor{red}{Duplication or speciation?}};
\draw [draw=red,->, line width=2pt]
      (dupspecLabel) to [bend right=15](AB.mid);
\draw [draw=red,->, line width=2pt]
      (dupspecLabel) to (ABC.mid);
%Slide 3: Example
\onslide<3>
\node [left=2cm of ABC,yshift=15mm](dupspecLabel){(Example)};
\node [left=5mm of AB,yshift=10mm](specLabel){\textcolor{ForestGreen}{Speciation}};
\node [right=1cm of ABC,yshift=5mm](dupLabel){\textcolor{Red}{Duplication}};
\draw [draw=ForestGreen,->, line width=2pt]
      (specLabel) to (AB.mid);
\draw [draw=red,->,line width=2pt]
      (dupLabel) to (ABC.mid);

\node [right=5mm of proteinC,text width=3cm,yshift=2cm] (txt)
      {B: \textit{ortholog}\\C: \textit{paralog}};

\end{tikzpicture}
\end{frame}

\begin{frame}{SIFTER}
\begin{itemize}
 \item \textit{Statistical Inference of Function Through Evolutionary Relationships}
 \item Predicts protein function (homology-based) given a reconciled tree\\
 \textrightarrow Tree construction \& conciliation remains a problem
 \item Based on bayesian statistics
 \item Complex statistics (not shown here)
\end{itemize}

\end{frame}

\begin{frame}[allowframebreaks]{Phylogenomic inference methodology}
 \begin{enumerate}
  \item Cluster homolog proteins
  \item Compute multiple alignment
  \item Edit alignment (remove potential non-homologs)
  \item Mask less-conserved regions in alignment
  \item Construct phylogenetic tree
  \item Identify closely related subtrees
  \item Overlay with experimental data
  \item Differentiate \textit{orthologs} and \textit{paralogs}\\
	(\textit{Tree reconciliation})
  \item Infer function from \textit{orthologs}
 \end{enumerate}
 \framebreak
\end{frame}

\begin{frame}[allowframebreaks]{Issues of phylogenomic methods}
  \textcolor{ForestGreen}{\textit{in-silico}}\ --\ \textcolor{Red}{Involves manual steps}
 \begin{enumerate}
  \item \textcolor{Red}{Cluster homolog proteins}
  \item \textcolor{ForestGreen}{Compute multiple alignment}
  \item \textcolor{Red}{Edit alignment}
  \item \textcolor{Red}{Mask less-conserved regions in alignment}
  \item \textcolor{ForestGreen}{Construct phylogenetic tree}
  \item \textcolor{ForestGreen}{Identify closely related subtrees}
  \item \textcolor{Red}{Overlay with experimental data}
  \item \textcolor{ForestGreen}{Differentiate \textit{orthologs} and \textit{paralogs}}
  \item \textcolor{Red}{Infer function from \textit{orthologs}}
 \end{enumerate}
 \framebreak
 \begin{enumerate}
  \item Cluster homolog proteins
  \item Compute multiple alignment
  \item Edit alignment (remove potential non-homologs)
  \item Mask less-conserved regions in alignment
  \end{enumerate}
  \begin{itemize}
   \item Manual annotation \& selection\\
   \textrightarrow Subjective, error-prone, time/cost-intensive
  \end{itemize}
 %%%
 \framebreak
 \begin{enumerate}
  \setcounter{enumi}{4}
  \item Construct phylogenetic tree
 \end{enumerate}
 \begin{itemize}
  \item \textit{Distance-based} vs. \textit{character-based} construction algorithms
  \item Small, highly-conserved protein families perform better than large (super)families
  \item Lack of consistency across methods
  \item Algorithms scale poorly
  \textrightarrow Can't be used for large (super)families
  \item Some methods produce millions of equivalently scored topologies
 \end{itemize}
 %%%
 \framebreak
 \begin{enumerate}
  \setcounter{enumi}{6}
  \item Overlay with experimental data
 \end{enumerate}
 \begin{itemize}
  \item Database = Experimental data + sinferred data
  \item Experimental datasets available
        $\leftrightarrow$ Protein function already know
  \item Protein function unknown $\leftrightarrow$ few experimental datasets available
 \end{itemize}
 %%%
 \framebreak
 \begin{itemize}
  \item Huge sets of parameters
  \item Requires manual annotation  experimental data
 \end{itemize}
\end{frame}

\begin{frame}[allowframebreaks]{Future of phylogenomic inference}
 \begin{itemize}
  \item Phylogenomics alone has too many problems and open questions, but...
  \pause
  \item ...\textbf{together with other concepts} functional prediction accuracy can be enhanced
  \item Computational complexity: Moore's law and alternative computational hardware\\
  \textrightarrow Large-scale application feasible in the future?
  \item 
 \end{itemize}
\end{frame}

 
\begin{frame}{Conclusion (Phylogenomic inference)}
 \begin{itemize}
  \item Powerful concept to separate \textit{orthologs} and \textit{paralogs}
  \pause
  \item ... if it would actually work in practice
  \item Too complex, too manual, too many parameters
  \item Pure \textit{in-silico} phylogenomics\\
  \textrightarrow\ Low quality results
  \item Manual annotation can't keep up with \textit{HTS}
 \end{itemize}
\end{frame}

\begin{frame}{Conclusion (Seminar)}
\begin{itemize}
\item Protein function inference is a yet unsolved problem in computational biology
\item Combine any information that is available, e.g.:
\begin{itemize}
 \item Phylogenomics
 \item Context-based prediction
 \item Pseudogenes
 \item Horizontal gene transfer
 \item Alternative splicing
 \item SNPs
 \item Experimental databases
\end{itemize}

\end{itemize}
\end{frame}


\begin{frame}
 \begin{center}
 {\color{BlueViolet}\large Thank you for your attention!}\\
 {\small References and sources available at}\\[1cm]
 {\small \url{https://github.com/ulikoehler/Hauptseminar}}
 \end{center}
\end{frame}

\end{document}